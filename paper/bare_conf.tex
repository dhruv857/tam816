
%% bare_conf.tex
%% V1.4b
%% 2015/08/26
%% by Michael Shell
%% See:
%% http://www.michaelshell.org/
%% for current contact information.
%%
%% This is a skeleton file demonstrating the use of IEEEtran.cls
%% (requires IEEEtran.cls version 1.8b or later) with an IEEE
%% conference paper.
%%
%% Support sites:
%% http://www.michaelshell.org/tex/ieeetran/
%% http://www.ctan.org/pkg/ieeetran
%% and
%% http://www.ieee.org/

%%*************************************************************************
%% Legal Notice:
%% This code is offered as-is without any warranty either expressed or
%% implied; without even the implied warranty of MERCHANTABILITY or
%% FITNESS FOR A PARTICULAR PURPOSE! 
%% User assumes all risk.
%% In no event shall the IEEE or any contributor to this code be liable for
%% any damages or losses, including, but not limited to, incidental,
%% consequential, or any other damages, resulting from the use or misuse
%% of any information contained here.
%%
%% All comments are the opinions of their respective authors and are not
%% necessarily endorsed by the IEEE.
%%
%% This work is distributed under the LaTeX Project Public License (LPPL)
%% ( http://www.latex-project.org/ ) version 1.3, and may be freely used,
%% distributed and modified. A copy of the LPPL, version 1.3, is included
%% in the base LaTeX documentation of all distributions of LaTeX released
%% 2003/12/01 or later.
%% Retain all contribution notices and credits.
%% ** Modified files should be clearly indicated as such, including  **
%% ** renaming them and changing author support contact information. **
%%*************************************************************************


% *** Authors should verify (and, if needed, correct) their LaTeX system  ***
% *** with the testflow diagnostic prior to trusting their LaTeX platform ***
% *** with production work. The IEEE's font choices and paper sizes can   ***
% *** trigger bugs that do not appear when using other class files.       ***                          ***
% The testflow support page is at:
% http://www.michaelshell.org/tex/testflow/


\documentclass[12pt,conference]{IEEEtran}
\usepackage{url}
\usepackage{graphicx}
\usepackage[stable]{footmisc}
\graphicspath{ {./images/} }
\usepackage{float}
\floatstyle{boxed} 
\restylefloat{figure}
\usepackage{physics}
\usepackage{cite}
\usepackage{fancyhdr}
% Some Computer Society conferences also require the compsoc mode option,
% but others use the standard conference format.
%
% If IEEEtran.cls has not been installed into the LaTeX system files,
% manually specify the path to it like:
% \documentclass[conference]{../sty/IEEEtran}





% Some very useful LaTeX packages include:
% (uncomment the ones you want to load)


% *** MISC UTILITY PACKAGES ***
%
%\usepackage{ifpdf}
% Heiko Oberdiek's ifpdf.sty is very useful if you need conditional
% compilation based on whether the output is pdf or dvi.
% usage:
% \ifpdf
%   % pdf code
% \else
%   % dvi code
% \fi
% The latest version of ifpdf.sty can be obtained from:
% http://www.ctan.org/pkg/ifpdf
% Also, note that IEEEtran.cls V1.7 and later provides a builtin
% \ifCLASSINFOpdf conditional that works the same way.
% When switching from latex to pdflatex and vice-versa, the compiler may
% have to be run twice to clear warning/error messages.






% *** CITATION PACKAGES ***
%
%\usepackage{cite}
% cite.sty was written by Donald Arseneau
% V1.6 and later of IEEEtran pre-defines the format of the cite.sty package
% \cite{} output to follow that of the IEEE. Loading the cite package will
% result in citation numbers being automatically sorted and properly
% "compressed/ranged". e.g., [1], [9], [2], [7], [5], [6] without using
% cite.sty will become [1], [2], [5]--[7], [9] using cite.sty. cite.sty's
% \cite will automatically add leading space, if needed. Use cite.sty's
% noadjust option (cite.sty V3.8 and later) if you want to turn this off
% such as if a citation ever needs to be enclosed in parenthesis.
% cite.sty is already installed on most LaTeX systems. Be sure and use
% version 5.0 (2009-03-20) and later if using hyperref.sty.
% The latest version can be obtained at:
% http://www.ctan.org/pkg/cite
% The documentation is contained in the cite.sty file itself.






% *** GRAPHICS RELATED PACKAGES ***
%
\ifCLASSINFOpdf
  % \usepackage[pdftex]{graphicx}
  % declare the path(s) where your graphic files are
  % \graphicspath{{../pdf/}{../jpeg/}}
  % and their extensions so you won't have to specify these with
  % every instance of \includegraphics
  % \DeclareGraphicsExtensions{.pdf,.jpeg,.png}
\else
  % or other class option (dvipsone, dvipdf, if not using dvips). graphicx
  % will default to the driver specified in the system graphics.cfg if no
  % driver is specified.
  % \usepackage[dvips]{graphicx}
  % declare the path(s) where your graphic files are
  % \graphicspath{{../eps/}}
  % and their extensions so you won't have to specify these with
  % every instance of \includegraphics
  % \DeclareGraphicsExtensions{.eps}
\fi
% graphicx was written by David Carlisle and Sebastian Rahtz. It is
% required if you want graphics, photos, etc. graphicx.sty is already
% installed on most LaTeX systems. The latest version and documentation
% can be obtained at: 
% http://www.ctan.org/pkg/graphicx
% Another good source of documentation is "Using Imported Graphics in
% LaTeX2e" by Keith Reckdahl which can be found at:
% http://www.ctan.org/pkg/epslatex
%
% latex, and pdflatex in dvi mode, support graphics in encapsulated
% postscript (.eps) format. pdflatex in pdf mode supports graphics
% in .pdf, .jpeg, .png and .mps (metapost) formats. Users should ensure
% that all non-photo figures use a vector format (.eps, .pdf, .mps) and
% not a bitmapped formats (.jpeg, .png). The IEEE frowns on bitmapped formats
% which can result in "jaggedy"/blurry rendering of lines and letters as
% well as large increases in file sizes.
%
% You can find documentation about the pdfTeX application at:
% http://www.tug.org/applications/pdftex





% *** MATH PACKAGES ***
%
%\usepackage{amsmath}
% A popular package from the American Mathematical Society that provides
% many useful and powerful commands for dealing with mathematics.
%
% Note that the amsmath package sets \interdisplaylinepenalty to 10000
% thus preventing page breaks from occurring within multiline equations. Use:
%\interdisplaylinepenalty=2500
% after loading amsmath to restore such page breaks as IEEEtran.cls normally
% does. amsmath.sty is already installed on most LaTeX systems. The latest
% version and documentation can be obtained at:
% http://www.ctan.org/pkg/amsmath





% *** SPECIALIZED LIST PACKAGES ***
%
%\usepackage{algorithmic}
% algorithmic.sty was written by Peter Williams and Rogerio Brito.
% This package provides an algorithmic environment fo describing algorithms.
% You can use the algorithmic environment in-text or within a figure
% environment to provide for a floating algorithm. Do NOT use the algorithm
% floating environment provided by algorithm.sty (by the same authors) or
% algorithm2e.sty (by Christophe Fiorio) as the IEEE does not use dedicated
% algorithm float types and packages that provide these will not provide
% correct IEEE style captions. The latest version and documentation of
% algorithmic.sty can be obtained at:
% http://www.ctan.org/pkg/algorithms
% Also of interest may be the (relatively newer and more customizable)
% algorithmicx.sty package by Szasz Janos:
% http://www.ctan.org/pkg/algorithmicx




% *** ALIGNMENT PACKAGES ***
%
%\usepackage{array}
% Frank Mittelbach's and David Carlisle's array.sty patches and improves
% the standard LaTeX2e array and tabular environments to provide better
% appearance and additional user controls. As the default LaTeX2e table
% generation code is lacking to the point of almost being broken with
% respect to the quality of the end results, all users are strongly
% advised to use an enhanced (at the very least that provided by array.sty)
% set of table tools. array.sty is already installed on most systems. The
% latest version and documentation can be obtained at:
% http://www.ctan.org/pkg/array


% IEEEtran contains the IEEEeqnarray family of commands that can be used to
% generate multiline equations as well as matrices, tables, etc., of high
% quality.




% *** SUBFIGURE PACKAGES ***
%\ifCLASSOPTIONcompsoc
%  \usepackage[caption=false,font=normalsize,labelfont=sf,textfont=sf]{subfig}
%\else
%  \usepackage[caption=false,font=footnotesize]{subfig}
%\fi
% subfig.sty, written by Steven Douglas Cochran, is the modern replacement
% for subfigure.sty, the latter of which is no longer maintained and is
% incompatible with some LaTeX packages including fixltx2e. However,
% subfig.sty requires and automatically loads Axel Sommerfeldt's caption.sty
% which will override IEEEtran.cls' handling of captions and this will result
% in non-IEEE style figure/table captions. To prevent this problem, be sure
% and invoke subfig.sty's "caption=false" package option (available since
% subfig.sty version 1.3, 2005/06/28) as this is will preserve IEEEtran.cls
% handling of captions.
% Note that the Computer Society format requires a larger sans serif font
% than the serif footnote size font used in traditional IEEE formatting
% and thus the need to invoke different subfig.sty package options depending
% on whether compsoc mode has been enabled.
%
% The latest version and documentation of subfig.sty can be obtained at:
% http://www.ctan.org/pkg/subfig




% *** FLOAT PACKAGES ***
%
%\usepackage{fixltx2e}
% fixltx2e, the successor to the earlier fix2col.sty, was written by
% Frank Mittelbach and David Carlisle. This package corrects a few problems
% in the LaTeX2e kernel, the most notable of which is that in current
% LaTeX2e releases, the ordering of single and double column floats is not
% guaranteed to be preserved. Thus, an unpatched LaTeX2e can allow a
% single column figure to be placed prior to an earlier double column
% figure.
% Be aware that LaTeX2e kernels dated 2015 and later have fixltx2e.sty's
% corrections already built into the system in which case a warning will
% be issued if an attempt is made to load fixltx2e.sty as it is no longer
% needed.
% The latest version and documentation can be found at:
% http://www.ctan.org/pkg/fixltx2e


%\usepackage{stfloats}
% stfloats.sty was written by Sigitas Tolusis. This package gives LaTeX2e
% the ability to do double column floats at the bottom of the page as well
% as the top. (e.g., "\begin{figure*}[!b]" is not normally possible in
% LaTeX2e). It also provides a command:
%\fnbelowfloat
% to enable the placement of footnotes below bottom floats (the standard
% LaTeX2e kernel puts them above bottom floats). This is an invasive package
% which rewrites many portions of the LaTeX2e float routines. It may not work
% with other packages that modify the LaTeX2e float routines. The latest
% version and documentation can be obtained at:
% http://www.ctan.org/pkg/stfloats
% Do not use the stfloats baselinefloat ability as the IEEE does not allow
% \baselineskip to stretch. Authors submitting work to the IEEE should note
% that the IEEE rarely uses double column equations and that authors should try
% to avoid such use. Do not be tempted to use the cuted.sty or midfloat.sty
% packages (also by Sigitas Tolusis) as the IEEE does not format its papers in
% such ways.
% Do not attempt to use stfloats with fixltx2e as they are incompatible.
% Instead, use Morten Hogholm'a dblfloatfix which combines the features
% of both fixltx2e and stfloats:
%
% \usepackage{dblfloatfix}
% The latest version can be found at:
% http://www.ctan.org/pkg/dblfloatfix




% *** PDF, URL AND HYPERLINK PACKAGES ***
%
%\usepackage{url}
% url.sty was written by Donald Arseneau. It provides better support for
% handling and breaking URLs. url.sty is already installed on most LaTeX
% systems. The latest version and documentation can be obtained at:
% http://www.ctan.org/pkg/url
% Basically, \url{my_url_here}.




% *** Do not adjust lengths that control margins, column widths, etc. ***
% *** Do not use packages that alter fonts (such as pslatex).         ***
% There should be no need to do such things with IEEEtran.cls V1.6 and later.
% (Unless specifically asked to do so by the journal or conference you plan
% to submit to, of course. )


% correct bad hyphenation here
% \hyphenation{op-tical net-works semi-conduc-tor}


% paragraph indentation
\setlength{\parindent}{2.5em}
\setlength{\parskip}{1em}

% disable word break
\tolerance=1
\emergencystretch=\maxdimen
\hyphenpenalty=10000
\hbadness=10000

\IEEEoverridecommandlockouts

\begin{document}
%
% paper title
% Titles are generally capitalized except for words such as a, an, and, as,
% at, but, by, for, in, nor, of, on, or, the, to and up, which are usually
% not capitalized unless they are the first or last word of the title.
% Linebreaks \\ can be used within to get better formatting as desired.
% Do not put math or special symbols in the title.
\title{Analyzing Open Source GitHub Repositories\\ Towards Technology Acceptance Model}


% author names and affiliations
% use a multiple column layout for up to three different
% affiliations
\author{\IEEEauthorblockN{Dhruvil Gandhi}
\IEEEauthorblockA{ Seidenberg School of Computer Science and Information Systems\\
Pace University, 
New York, USA\\
Email: dgandhi@pace.edu}
}

% conference papers do not typically use \thanks and this command
% is locked out in conference mode. If really needed, such as for
% the acknowledgment of grants, issue a \IEEEoverridecommandlockouts
% after \documentclass

% for over three affiliations, or if they all won't fit within the width
% of the page, use this alternative format:
% 
%\author{\IEEEauthorblockN{Michael Shell\IEEEauthorrefmark{1},
%Homer Simpson\IEEEauthorrefmark{2},
%James Kirk\IEEEauthorrefmark{3}, 
%Montgomery Scott\IEEEauthorrefmark{3} and
%Eldon Tyrell\IEEEauthorrefmark{4}}
%\IEEEauthorblockA{\IEEEauthorrefmark{1}School of Electrical and Computer Engineering\\
%Georgia Institute of Technology,
%Atlanta, Georgia 30332--0250\\ Email: see http://www.michaelshell.org/contact.html}
%\IEEEauthorblockA{\IEEEauthorrefmark{2}Twentieth Century Fox, Springfield, USA\\
%Email: homer@thesimpsons.com}
%\IEEEauthorblockA{\IEEEauthorrefmark{3}Starfleet Academy, San Francisco, California 96678-2391\\
%Telephone: (800) 555--1212, Fax: (888) 555--1212}
%\IEEEauthorblockA{\IEEEauthorrefmark{4}Tyrell Inc., 123 Replicant Street, Los Angeles, California 90210--4321}}




% use for special paper notices
% \IEEEspecialpapernotice{(Final Submission)}




% make the title area
\maketitle

% As a general rule, do not put math, special symbols or citations
% in the abstract
\begin{abstract}
Open Source is a gift and effort of the technology community, over years, creating and supporting platforms, projects, etc. This paper analyzes data from GitHub public repositories available on Google's BigQuery and observes relations, trends and anomalies. To set a baseline, public repositories for 22 different languages are analyzed, correlations, anomalies and trends are studied.  
\end{abstract}

\begin{IEEEkeywords}
  Technology Acceptance Model, Data Analytics, Time Series Anomalies, Data Exploration
\end{IEEEkeywords}






% For peer review papers, you can put extra information on the cover
% page as needed:
% \ifCLASSOPTIONpeerreview
% \begin{center} \bfseries EDICS Category: 3-BBND \end{center}
% \fi
%
% For peerreview papers, this IEEEtran command inserts a page break and
% creates the second title. It will be ignored for other modes.



\section{Introduction}
% no \IEEEPARstart
% This demo file is intended to serve as a ``starter file''
% for IEEE conference papers produced under \LaTeX\ using
% IEEEtran.cls version 1.8b and later.\cite{dcagh}
% % You must have at least 2 lines in the paragraph with the drop letter
% % (should never be an issue)
% I wish you the best of success.
Open Source allows community from novice to advanced, everyone to collaborate and contribute to projects, platforms, programming languages and frameworks. Over time the advancement of languages and platforms; languages from Assembly Language to C to Python to TypeScript and Kotlin, for platforms - from Batch systems to multiprocessing and modern distributed systems. With the evolution of services and platforms, IaaS, Saas and PaaS, lead to the growth of open-source community and projects overtime. 

Various studies have been conducted overtime to study the impact and growth of these. This study \cite{ghinflu} studies the influence of software by studying GitHub. Evaluation \cite{tam1} and \cite{tamis} analysis observes and introduces Technology Acceptance Model, and develops a hypothesis for it. Another such study \cite{taagh} was conducted to study UTAUT (Unified Theory of Acceptance and Use of Technology)\cite{taagh} where a study was done on 78 undergraduate students for acceptance and impact of GitHub on their education. Various studies \cite{dcagh} \cite{nandi2016anomaly} have been performed on commit messages, coding style, pull requests, coding times and other factors available for the dataset from GitHub \cite{ght}

The trends of open source repositories on GitHub was studied and trends were established and anomalies were studied using ELK stack and BigQuery. Initially, correlation of trends and data with StackOverflow, a technology Q/A platform was proposed, which however is moved as next steps and future actions for this study. 

Methodology section talks about the technology used, dataset and analysis proposed. System and Experiment sections elaborate on the system setup used for the study and present the steps, algorithms, and data analytics methods used in the study. Results, dataset snippets, small data frames, and anomalies are discussed in the observation and result section. The paper ends with a preliminary conclusion and states the next steps and opportunities for analysis.


% \section{Literature Review}
% Subsection text here.

\section{Methodology}

A public dataset of nearly 3.8 million GitHub open source repositories\cite{gcp} is available on Google Cloud Platform's BigQuery service. This dataset was used in the study. BigQuery is Google's columnar storage service for high performance and high throughput needs. 

To store and analyze data, ElasticSearch and Kibana were used. ElasticSearch is based on Lucene which is a full-text search based search engine. ElasticSearch has advantage of indexing JSON documents without reindexing, only adding or updating the new or updated documents. To analyze and visualize, Kibana was used, Kibana is a web-based UI for ElasticSearch, which provides analysis and other services for ElasticSearch. 

To perform the study, data about repositories for following languages was loaded in ElasticSearch from BigQuery, using Python. The repositories for which the repositories were loaded are highlighted in table \ref{table:1}. The fields that were loaded from the BigQuery dataset are as in table \ref{table:2}. 


\begin{table}[h!]
  \centering
  \begin{tabular}{||c | c||} 
  \hline
  Language Name & Total Repositories  \\ [0.5ex] 
  \hline \hline
  C & 5,451,310 \\  \hline 
 Python & 878,748 \\  \hline 
 Go & 734,652 \\  \hline 
 C++ & 689,476 \\  \hline 
 Java & 626,740 \\  \hline 
 Dockerfile & 264,042 \\  \hline 
 Objective-C & 144,206 \\  \hline 
 Haskell & 52,430 \\  \hline 
 Clojure & 43,686 \\  \hline 
 Rust & 34,519 \\  \hline 
 R & 29,813 \\  \hline 
 Jupyter Notebook & 26,891 \\  \hline 
 Erlang & 25,694 \\  \hline 
 TypeScript & 14,372 \\  \hline 
 Julia & 13,529 \\  \hline 
 Kotlin & 12,915 \\  \hline 
 Swift & 9,379 \\  \hline 
 Objective-C++ & 60 \\  \hline 
 Vue & 38 \\  \hline 
 SQL & 20 \\ [1ex] 
  \hline
  \end{tabular}
\caption{Languages and Repositories}
\label{table:1}
\end{table}

\begin{table}[h!]
  \centering
  \begin{tabular}{||c | c||} 
  \hline
  Field Name & Data Type  \\ [0.5ex] 
  \hline \hline
 Repository Name & String \\ \hline
 Language Name & String \\ \hline
 License Name & String \\ \hline
 Timestamp of creation & Epoch Time in seconds\\ \hline
 Year of creation & Integer \\ \hline
 Month of creation & Integer \\ [0.5ex] 
  \hline
  \end{tabular}
\caption{Schema of data used}
\label{table:2}
\end{table}

To analyze, word cloud, average repositories overtime for all languages, different visualizations, time-series analysis and timeseries anomaly detection and forecast were carried out. The result is described in section V and additional materials supplied with the paper. 

One limitation while conducting the study is limit of availability of data on percentage of language for a particular repository, for example a simple webpage might have HTML and CSS, but it does not have data on what percent of each language is present in a particular repository.

\section{System and Experiment}

The project was first setup to load data into Azure cloud, but due to network limit because of the tier limit and different networks/datacenters, the data load was taking a long time. To speed up data loading, virtual machines that hosted ELK stack or ElasticSearch, LogStash and Kibana, was set up on Google Cloud Platform's Virtual Machine service. 

As laid out in the diagram, the data for languages of different public repositories as specified in table \ref{table:1} was loaded into BigQuery. This was done to take subset of data and not cross processing threshold of the GCP. Then using BigQuery's API and Python script, data as laid out in table \ref{table:2} was loaded into ElasticSearch's index. All the fields were set to searchable to enable them for analysis. This helped in analyzing on different factors. Once the data was loaded, different visualizations were made as seen in section IV, and a time-series anomaly machine learning \cite{elastic} job was started using Kibana for ElasticSearch. The analysis was done for repositories from 2009 to 2019. All the services, ElasticSearch, Kibana and Python scripts are currently hosted on 2.5GiB RAM, 4 vCPUs virtual instance on GCP.

After anomalies were highlighted, causes were searched on internet for possible causes. For three instances, for three different languages, they were linked with a conference or a major version release.  All the analysis, the code used and the reports obtained are stored on GitHub.

\floatstyle{boxed}
\restylefloat{figure}
\begin{figure*}[ht]
  \centering
  \includegraphics[width=16cm]{tsanomaly.png}
  \caption{Language Word Cloud by frequency}
  \label{fig:anomaly}
\end{figure*}

\floatstyle{boxed}
\restylefloat{figure}
\begin{figure*}[ht]
  \centering
  \includegraphics[width=16cm]{golaunchyear.png}
  \caption{Go language launch year stats}
  \label{fig:gyl}
\end{figure*}
% \clearpage

\section{Observation and Result}

Figure \ref{fig:anomaly} shows anomalies of 6 languages in red, when compared to news or significant event, it maps around the month of occurence of that event. For instance, Kotlin has a high spike, which when searched online, maps to beginning of involvement of a company called Instil, which consults and trains for Kotlin. Another example is for Python, we see a surge around 2014, a spike is observed which can be related to the announcement of Python 2.7 support end date which is end of 2020.

Figure \ref{fig:gyl} shows stats from 2015, which can be mapped to GopherCon in July, which is the reason for spiked activity in 7th month or July in 2015. 

Basic statistics about the data used in analyzing all the repositories is presented in table \ref{table:1} and \ref{table:2}. Figure \ref{fig:torl} shows the timestamp range of all the repositories loaded. 

\floatstyle{boxed}
\restylefloat{figure}
\begin{figure*}[ht]
  \centering
  \includegraphics[width=16cm]{actualvspredictedR.png}
  \caption{R forecast and anomaly prediction}
  \label{fig:rfc}
\end{figure*}

\floatstyle{boxed}
\restylefloat{figure}
\begin{figure*}[ht]
  \centering
  \includegraphics[width=16cm]{anomalygo.png}
  \caption{Go forecast and anomaly prediction}
  \label{fig:gfc}
\end{figure*}

\floatstyle{plain}
\restylefloat{figure}
\begin{figure}[h]
  \centering
  \includegraphics[width=8cm]{tsteposdata.png}
  \caption{Timestamp of repositories loaded}
  \label{fig:torl}
\end{figure}

Figure \ref{fig:lang} and figure \ref{fig:licenses} shows word cloud for number of repositories by language frequency and licenses frequency. 

\floatstyle{plain}
\restylefloat{figure}
\begin{figure}[h]
  \centering
  \includegraphics[width=8cm]{lang.png}
  \caption{Wordcloud of repositories loaded}
  \label{fig:lang}
\end{figure}

\floatstyle{plain}
\restylefloat{figure}
\begin{figure}[h!]
  \centering
  \includegraphics[width=8cm]{lic.png}
  \caption{Wordcloud of repositories loaded}
  \label{fig:licenses}
\end{figure}

Figure \ref{fig:mlj} shows the stats for anomaly machine learning job created and processed on elasticsearch. This job detected the results shows in \ref{fig:anomaly}. This job as stated earlier, was expanded to forecast, on multiple different languages and different time periods, the limit being 365 days due to memory limit for JVM and the virtual machine on GCP. This job predicted the trend in number of repositories over time, absolute accuracy was not calculated. However, by visual observation, it is closely following the historical trend for the examined languages. 

Figure \ref{fig:rfc} shows running R language single time-series anomaly detection job which was scheduled to predict the trend over a period of 5 years, and from the graph, blue being actual and yellow being forecasted, the trend follows similar line. Figure \ref{fig:gfc} show the same for Go language, however, this was on a shorter period of time, due to memory and resource restrictions on GCP's virtual instance. 

\floatstyle{plain}
\restylefloat{figure}
\begin{figure}[h!]
  \centering
  \includegraphics[width=8cm]{mljob.png}
  \caption{Anomaly machine learning job}
  \label{fig:mlj}
\end{figure}


\section{Future Work}
Due to limited time and resources, full analysis and StackOverflow's data loading and its analysis was not performed. Further extension of this research can be performed using data for following points:
\begin{itemize}
  \item StackOverflow data for language questions and answers, its time and sentiment analysis.
  \item GitHub commit messages, time, releases, pull requests, code comments and forking analysis.
  \item Correlating both datasets with mentions of repositories, issues and links.
  \item Getting or creating a dataset for significant events for a subset of programming language.
\end{itemize}
more parameters and data from stack and mention of repositories in stack

\section{Conclusion}
Throughout the paper, a pattern and trend has been observed for languages and how different significant event impact quantity of open source GitHub repositories. The time-series analysis helps to confirm the same. An algorithm to correlate trend with significant event occurence has not been devised. The anomalies, however, help detect outliers and relate them to a version release, support announcement or end of support announcement. 


% An example of a floating figure using the graphicx package.
% Note that \label must occur AFTER (or within) \caption.
% For figures, \caption should occur after the \includegraphics.
% Note that IEEEtran v1.7 and later has special internal code that
% is designed to preserve the operation of \label within \caption
% even when the captionsoff option is in effect. However, because
% of issues like this, it may be the safest practice to put all your
% \label just after \caption rather than within \caption{}.
%
% Reminder: the "draftcls" or "draftclsnofoot", not "draft", class
% option should be used if it is desired that the figures are to be
% displayed while in draft mode.
%
%\begin{figure}[!t]
%\centering
%\includegraphics[width=2.5in]{myfigure}
% where an .eps filename suffix will be assumed under latex, 
% and a .pdf suffix will be assumed for pdflatex; or what has been declared
% via \DeclareGraphicsExtensions.
%\caption{Simulation results for the network.}
%\label{fig_sim}
%\end{figure}

% Note that the IEEE typically puts floats only at the top, even when this
% results in a large percentage of a column being occupied by floats.


% An example of a double column floating figure using two subfigures.
% (The subfig.sty package must be loaded for this to work.)
% The subfigure \label commands are set within each subfloat command,
% and the \label for the overall figure must come after \caption.
% \hfil is used as a separator to get equal spacing.
% Watch out that the combined width of all the subfigures on a 
% line do not exceed the text width or a line break will occur.
%
%\begin{figure*}[!t]
%\centering
%\subfloat[Case I]{\includegraphics[width=2.5in]{box}%
%\label{fig_first_case}}
%\hfil
%\subfloat[Case II]{\includegraphics[width=2.5in]{box}%
%\label{fig_second_case}}
%\caption{Simulation results for the network.}
%\label{fig_sim}
%\end{figure*}
%
% Note that often IEEE papers with subfigures do not employ subfigure
% captions (using the optional argument to \subfloat[]), but instead will
% reference/describe all of them (a), (b), etc., within the main caption.
% Be aware that for subfig.sty to generate the (a), (b), etc., subfigure
% labels, the optional argument to \subfloat must be present. If a
% subcaption is not desired, just leave its contents blank,
% e.g., \subfloat[].


% An example of a floating table. Note that, for IEEE style tables, the
% \caption command should come BEFORE the table and, given that table
% captions serve much like titles, are usually capitalized except for words
% such as a, an, and, as, at, but, by, for, in, nor, of, on, or, the, to
% and up, which are usually not capitalized unless they are the first or
% last word of the caption. Table text will default to \footnotesize as
% the IEEE normally uses this smaller font for tables.
% The \label must come after \caption as always.
%
%\begin{table}[!t]
%% increase table row spacing, adjust to taste
%\renewcommand{\arraystretch}{1.3}
% if using array.sty, it might be a good idea to tweak the value of
% \extrarowheight as needed to properly center the text within the cells
%\caption{An Example of a Table}
%\label{table_example}
%\centering
%% Some packages, such as MDW tools, offer better commands for making tables
%% than the plain LaTeX2e tabular which is used here.
%\begin{tabular}{|c||c|}
%\hline
%One & Two\\
%\hline
%Three & Four\\
%\hline
%\end{tabular}
%\end{table}


% Note that the IEEE does not put floats in the very first column
% - or typically anywhere on the first page for that matter. Also,
% in-text middle ("here") positioning is typically not used, but it
% is allowed and encouraged for Computer Society conferences (but
% not Computer Society journals). Most IEEE journals/conferences use
% top floats exclusively. 
% Note that, LaTeX2e, unlike IEEE journals/conferences, places
% footnotes above bottom floats. This can be corrected via the
% \fnbelowfloat command of the stfloats package.



% conference papers do not normally have an appendix


% use section* for acknowledgment
\section*{Acknowledgment}

I would like to thank Dr. Charles Tappert, Dr. Ronald Frank and Professor Stephan Barabasi for mentoring and guidance. I would also like to thank the Elastic.co team to extend my license twice which allowed me to use all their features and perform analysis. 

% trigger a \newpage just before the given reference
% number - used to balance the columns on the last page
% adjust value as needed - may need to be readjusted if
% the document is modified later
%\IEEEtriggeratref{8}
% The "triggered" command can be changed if desired:
%\IEEEtriggercmd{\enlargethispage{-5in}}

% references section

% can use a bibliography generated by BibTeX as a .bbl file
% BibTeX documentation can be easily obtained at:
% http://mirror.ctan.org/biblio/bibtex/contrib/doc/
% The IEEEtran BibTeX style support page is at:
% http://www.michaelshell.org/tex/ieeetran/bibtex/
\bibliographystyle{IEEEtran}
\bibliography{refs}




% that's all folks
\end{document}

